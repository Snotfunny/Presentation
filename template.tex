%%%%%%%%%%%%%%%%%%%%%%%%%%%%%%%%%%%%%%%%%
% Focus Beamer Presentation
% LaTeX Template
% Version 1.0 (8/8/18)
%
% This template has been downloaded from:
% http://www.LaTeXTemplates.com
%
% Original author:
% Pasquale Africa (https://github.com/elauksap/focus-beamertheme) with modifications by 
% Vel (vel@LaTeXTemplates.com)
%
% Template license:
% GNU GPL v3.0 License
%
% Important note:
% The bibliography/references need to be compiled with bibtex.
%
%%%%%%%%%%%%%%%%%%%%%%%%%%%%%%%%%%%%%%%%%

%----------------------------------------------------------------------------------------
%	PACKAGES AND OTHER DOCUMENT CONFIGURATIONS
%----------------------------------------------------------------------------------------

\documentclass{beamer}

\usetheme{focus} % Use the Focus theme supplied with the template
% Add option [numbering=none] to disable the footer progress bar
% Add option [numbering=fullbar] to show the footer progress bar as always full with a slide count

% Uncomment to enable the ice-blue theme
\definecolor{main}{RGB}{82, 32, 76}
\definecolor{background}{RGB}{255, 255, 255}

%------------------------------------------------

\usepackage{booktabs} % Required for better table rules

\usepackage{amsthm,amsfonts,amsmath,amssymb}
\usepackage{graphicx}
\graphicspath{{figures/}}

%----------------------------------------------------------------------------------------
%	 TITLE SLIDE
%----------------------------------------------------------------------------------------

\title{Left-Orderable Groups}

\author{James Roe}


%------------------------------------------------

\begin{document}

%------------------------------------------------

\begin{frame}
	\maketitle % Automatically created using the information in the commands above
\end{frame}

%----------------------------------------------------------------------------------------
%	 Overview
%-------------------------------------------------------

\begin{frame}{Overview}
	\begin{enumerate}
		\item What is a Left-Orderable Group?
		\item The Positive Cone
		\item An Example: $\mathbb{Z}^2$
		\item Free Groups
		\item Why do we care?
	\end{enumerate}
\end{frame}

%--------------------------------------------------------------------
%LO Groups
%---------------------------------------------------------------------

\begin{frame}{What is a Left-Orderable Group?}
\begin{block}{Strict Orderings}
	Let $G$ be a group. $<$ is a \textit{strict, total ordering} on $G$ if it is:
	
	i) Transitive ($g<h, g<h \Rightarrow g<h$)
	
	ii) $\forall g,h \in G$, exactly one of $g<h$, $h<g$ or $g=h$ holds.  
\end{block}
\pause
\begin{block}{Invariant Orderings}
	$G$ is \textit{left-ordered} if $<$ is \textit{left-invariant}, i.e.
	\[
	\forall g,h,f \in G, g<h \Rightarrow fg<fh
	\]  
\end{block}
\end{frame}

%------------------------------------------------------------
%Positive Cone
%----------------------------------------------------------------
\begin{frame}{The Positive Cone}
\begin{itemize}
	\item $P \subset G$ characterizes the ordering on $G$. 
	\[
	g \in P \iff 1<g
	\]
	\pause
	\item $P$ must satisfy:
	
	i) $P$ is a subsemigroup (i.e. $g,h \in P \Rightarrow gh \in P$) 
	
	ii) $P \cup P^{-1} = P \backslash \{1\}$
	
	iii) $P \cap P^{-1} = \emptyset$
\end{itemize}
\end{frame}

%------------------------------------------------
%Z2
%--------------------------------------------------

\begin{frame}{An Example: $\mathbb{Z}^2$}
	\begin{columns}
		\column{0.5\textwidth}
		\begin{itemize}
		\item $v$ a vector in $\mathbb{R}^2$ with irrational gradient.
		
		\item $\underline{x}<\underline{y} \iff \underline{x} \cdot v < \underline{y} \cdot v$\
		\end{itemize}
		\column{0.5\textwidth}
		\includegraphics[width=\linewidth]{Z2.png}
	\end{columns}
\end{frame}

%------------------------------------------------
%Z2 2
%----------------------------------------------------------
\begin{frame}{An Example: $\mathbb{Z}^2$}
\begin{columns}
	\column{0.5\textwidth}
	\begin{itemize}
	\item $v$ a vector in $\mathbb{R}^2$ with irrational gradient.
	
	\item $\underline{x}<\underline{y} \iff \underline{x} \cdot v < \underline{y} \cdot v$\

	\item $P$ is one side of $l' \bot l$.
	\end{itemize}
	\column{0.5\textwidth}
	\includegraphics[width=\linewidth]{Z2Poscone.png}
\end{columns}
\end{frame}
%---------------------------------------------------------------
%---------------------------------------------------------
%-------------------------------------------------------

\begin{frame}[t]
	This slide has an empty title and is aligned to top.
\end{frame}

%------------------------------------------------

\begin{frame}[noframenumbering]{No Slide Numbering}
	This slide is not numbered and is citing reference \cite{knuth74}.
\end{frame}

%------------------------------------------------

\begin{frame}{Typesetting and Math}
	The packages \texttt{inputenc} and \texttt{FiraSans}\footnote{\url{https://fonts.google.com/specimen/Fira+Sans}}\textsuperscript{,}\footnote{\url{http://mozilla.github.io/Fira/}} are used to properly set the main fonts.
	\vfill
	This theme provides styling commands to typeset \emph{emphasized}, \alert{alerted}, \textbf{bold}, \textcolor{example}{example text}, \dots
	\vfill
	\texttt{FiraSans} also provides support for mathematical symbols:
	\begin{equation*}
		e^{i\pi} + 1 = 0.
	\end{equation*}
\end{frame}

%----------------------------------------------------------------------------------------
%	 SECTION 2
%----------------------------------------------------------------------------------------

\section{Section 2}

%------------------------------------------------

\begin{frame}{Blocks}
	These blocks are part of 1 slide, to be displayed consecutively.
	\begin{block}{Block}
		Text.
	\end{block}
	\pause % Automatically creates a new "page" split between the above and above + below
	\begin{alertblock}{Alert block}
		Alert \alert{text}.
	\end{alertblock}
	\pause % Automatically creates a new "page" split between the above and above + below
	\begin{exampleblock}{Example block}
		Example \textcolor{example}{text}.
	\end{exampleblock}
\end{frame}

%------------------------------------------------

\begin{frame}{Columns}
	\begin{columns}
		\column{0.5\textwidth}
			This text appears in the left column and wraps neatly with a margin between columns.
		
		\column{0.5\textwidth}
			\includegraphics[width=\linewidth]{Images/placeholder.jpg}
	\end{columns}
\end{frame}

%------------------------------------------------

\begin{frame}{Lists}
	\begin{columns}[T, onlytextwidth] % T for top align, onlytextwidth to suppress the margin between columns
		\column{0.33\textwidth}
			Items:
			\begin{itemize}
				\item Item 1
				\begin{itemize}
					\item Subitem 1.1
					\item Subitem 1.2
				\end{itemize}
				\item Item 2
				\item Item 3
			\end{itemize}
		
		\column{0.33\textwidth}
			Enumerations:
			\begin{enumerate}
				\item First
				\item Second
				\begin{enumerate}
					\item Sub-first
					\item Sub-second
				\end{enumerate}
				\item Third
			\end{enumerate}
		
		\column{0.33\textwidth}
			Descriptions:
			\begin{description}
				\item[First] Yes.
				\item[Second] No.
			\end{description}
	\end{columns}
\end{frame}

%------------------------------------------------

\begin{frame}{Table}
	\begin{table}
		\centering % Centre the table on the slide
		\begin{tabular}{l c}
			\toprule
			Discipline & Avg. Salary \\
			\toprule
			\textbf{Engineering} & \textbf{\$66,521} \\
			Computer Sciences & \$60,005\\
			Mathematics and Sciences & \$61,867\\
			Business & \$56,720\\
			Humanities \& Social Sciences & \$56,669\\
			Agriculture and Natural Resources & \$53,565\\
			Communications & \$51,448\\
			\midrule
			\textbf{Average for All Disciplines} & \textbf{\$58,114}\\
			\bottomrule
		\end{tabular}
	\caption{Table caption}
	\end{table}
\end{frame}

%------------------------------------------------

\begin{frame}[focus]
	Thanks for using \textbf{Focus}!
\end{frame}

%----------------------------------------------------------------------------------------
%	 CLOSING/SUPPLEMENTARY SLIDES
%----------------------------------------------------------------------------------------

\appendix

\begin{frame}{References}
	\nocite{*} % Display all references regardless of if they were cited
	\bibliography{example.bib}
	\bibliographystyle{plain}
\end{frame}

%------------------------------------------------

\begin{frame}{Backup Slide}
	This is a backup slide, useful to include additional materials to answer questions from the audience.
	\vfill
	The package \texttt{appendixnumberbeamer} is used to refrain from numbering appendix slides.
\end{frame}

%----------------------------------------------------------------------------------------

\end{document}
